\phantomsection\section*{Выводы}\addcontentsline{toc}{section}{Выводы}

Функции обработчика прерывания от системного таймера для операционных систем семейства UNIX и для семейства Windows схожи, так как они являются системами разделения времени.

 Схожие задачи обработчика прерывания от системного таймера:
\begin{itemize}
	\item инициализируют отложенные действия, относящиеся к работе планировщика, такие как пересчёт приоритетов;
	\item выполняют декремент счётчиков времени: часов, таймеров, будильников реального времени, счётчиков времени отложенных действий;
	\item выполняют декремент кванта: текущего процесса в UNIX, текущего потока в Windows.
\end{itemize}

Пересчет динамических приоритетов осуществляется только для пользовательских процессов, чтобы избежать бесконечного откладывания. Обе операционные системы (UNIX и Windows) -- это системы разделения времени с динамическими приоритетами и вытеснением.

В UNIX приоритет пользовательского процесса (процесса в режиме задач) может динамически пересчитываться, в зависимости от трех факторов. Приоритеты ядра -- фиксированные величины.

В Windows при создании процесса ему назначается базовый приоритет, относительно базового приоритета процесса потоку назначается относительный приоритет, таким образом, у потока нет своего приоритета. Приоритет потока пользовательского процесса может быть динамически пересчитан.