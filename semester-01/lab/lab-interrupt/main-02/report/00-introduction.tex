\phantomsection\section*{Введение}\addcontentsline{toc}{section}{Введение}

Таймер -- это аппаратное устройство, которое генерирует периодические прерывания с фиксированным интервалом времени. В классических UNIX--системах этот интервал составляет 10 мс и называется тиком. 

\textbf{Тик} -- это время между двумя последовательными прерываниями таймера.

\textbf{Главный тик} -- это временной отрезок, равный $N$--тикам таймера,  где $N$ -- системозависимая величина, определяемая операционной системой.

\textbf{Квант} -- это временной отрезок, выделяемый процессу планировщиком для выполнения, то есть в течение которого процесс может использовать процессор.